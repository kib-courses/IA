\documentclass[english,russian,12pt]{article}
\usepackage[T1,T2A]{fontenc}
\usepackage[utf8]{inputenc}
\usepackage{babel}

\title{Введение в современную архитектуру Intel}
\date{Сентябрь\\ 2020}
\author{Горячий Максим Сергеевич}
\begin{document}
\maketitle

\subsection*{Описание курса}
В курсе будут рассмотрены современные возможности архитектуры Intel.


\subsection*{Содержание курса}
\begin{enumerate}
  \item Обзор архитектуры IA-32/32e. Регистры и память. Регистры общего назначения, регистры x87/MMX/SSE/AVX/AVX-2/AVX-512, сегментные регистры, моделезависимые регистры (MSRs), модель памяти, адресное пространство ввода-вывода (I/O), configuration spaces.
  \item Система команд. Команды общего назначения, команды математического сопроцессора, MMX, SSE, AVX, AVX-2, AVX-512. Команды передачи управления. Специализированные команды.
  \item Режимы работы. Реальный режим (Real Mode). Защищённый режим (Protected Mode).  Virtual-8086. System Management Mode. IA32e (Long Mode, режим совместимости).
  \item Защищённый режим. Уровни привилегий. Сегментные дескрипторы. Сегментные регистры (CS, SS, DS, ES, FS, GS). GDT. LDT. Плоская модель памяти. Сегментация в Long Mode.
  \item Кеш. UC/WC/WP/WT/WB типы памяти. MTRRs. PAT.
  \item Paging. Структуры PDE, PTE. TLB. PAE. PDPE. IA-32e (PLM4E). Механизмы обеспечения безопасности: NX bit, SMAP, SMEP.
  \item Обзор UEFI. Стадии загрузки. PEI и DXE драйверы. EDK2.
  \item Поддержка многозадачности.
  \item Передача управления. Call Gates. SYSCALL. SYSENTER. Автоматическое приключение стека. TSS. Task Gates.
  \item Прерывания и исключения. Аппаратные и программные прерывания. Приоритет прерываний. APIC. Local APIC. Поддержка многопроцессорности.
  \item Поддержка аппаратной виртуализации (Intel-VTx). VMX Root Mode, Guest Mode. Virtual Machine Control Structure. VMLAUNCH, VMRESUME, VMEXIT. Shadow Page Tables. Extended (Nested) Page Tables. TLB Management with Virtualization.
  \item {\it Обзор расширений для защиты данных. Intel SGX. Intel MPX.}
  \item {\it Режим System Management Mode.}
\end{enumerate}

\subsection*{Итоговые знания}
\begin{itemize}
  \item Базовое понимание архитектуры x86: регистры, память, режимы работы.
  \item Принципы работы виртуальной памяти в различных режимах работы процессора.
  \item Базовое понимание архитектуры UEFI, навыки разработки UEFI модулей.
  \item Что такое аппаратная визуализация и современные механизмы защиты.
\end{itemize}

\nocite{SDMVol1}
\nocite{SDMVol2}
\nocite{SDMVol3}
\nocite{SDMVol4}
\nocite{Optim}
\nocite{Fog}
\nocite{InstSet}
\nocite{Pentium}
\nocite{Guk8086}
\nocite{GukPen3}

\bibliographystyle{unsrt}
\bibliography{program}

\end{document} 